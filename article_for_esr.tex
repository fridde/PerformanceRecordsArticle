
% ------------------------------------------------------------------------------
% Things that have to be fixed before final version:
% In bib-file:
% - Replace \{ with { and \} with }
% - Replace foreign characters, ��� etc
% - Replace howpublished with bla or any other nonsense word to avoid duplicated urls
%  In tex-file:
% - replace \citep{ with \citep{
% - uncomment \usepackage{hyperref}
% ------------------------------------------------------------------------------
\documentclass[final,authoryear,1p]{elsarticle}


\usepackage{amsmath}
\usepackage{amssymb}
\usepackage[latin1]{inputenc}
\usepackage[hyphens]{url}
\usepackage{graphicx}
\usepackage{caption}
\usepackage{wasysym}
\usepackage{booktabs}
\usepackage{subcaption}
\usepackage[colorlinks=true, pdfborder={0 0 0}]{hyperref}
\usepackage{pdfpages}

\widowpenalty=10000



\biboptions{square}

\journal{Uppsala University}

\begin{document}

\begin{frontmatter}

\title{Establishing performance records for prognoses using Canadian oil sands as example}

%% use optional labels to link authors explicitly to addresses:
%% \author[label1,label2]{<author name>}
%% \address[label1]{<address>}
%% \address[label2]{<address>}

\author{Friedrich Hehl}

\date{\today}

\begin{abstract}
%% Text of abstract

\textbf{\emph{Note: Old abstract, taken from Master Thesis}}

Crude oil plays an important role for the global energy system. As there is ample evidence that conventional oil production will have peaked by 2020, unconventional oil has attained a stronger focus. In particular, oil derived from bitumen from Canadian oil sands has been proposed as a possible remedy to global oil depletion. 

This study aims to test the hypothesis that forecasts on the Canadian oil sands published between about 2000 and 2010 have been overestimating production significantly.
A large compilation of oil sands projects, prognoses and production data has been established using openly available databases and reports. Conversion, standardization and analysis of the data were done using a webinterface made solely for this purpose.

The statistical analysis shows a significant bias of the prognoses towards an overestimation of oil sands production. The compilation shows that most authors tend to overestimate the rate of expansion of the industry. Therefore, any prognosis on the expansion of the industry should be examined thoroughly before use.
\end{abstract}

\begin{keyword}
oil sands \sep unconventional oil \sep analysis \sep heavy oil 
%% keywords here, in the form: keyword \sep keyword

\end{keyword}

\end{frontmatter}


\clearpage
% \linenumbers

\section{Introduction}
\label{sec:Introduction}
Oil produced from Canadian oil sands has been pointed out as a possible remedy to mitigate a future decline of conventional oil production. 
Since the so-called \emph{Hirsch-report} \cite{Hirsch2005}, which proposed the further expansion of the industry as a means to sustain global economic welfare, numerous articles, publications and comments have been submitted to the public for evaluation.  
Among these are several outlooks and prognoses for the future development of the oil sands industry.

However, these prognoses  are rarely made accountable for their performance compared to the actual course of events later on. 
Since energy strategies heavily rely on the choice of the prognosis that supports the strategy, informed choices should evaluate the performance of earlier prognoses made with the same model or, if a model is not disclosed, the same institution.

Surprisingly, performance ratings of historical prognoses have not been established as a standard tool when choosing a model, and, in fact, do not exist for most of the models.

This article evaluates the performance of several prognoses made on the development of the production from Canadian oil sands made between 2000 and 2012.  
It shows that prognoses have tended to overestimate production vastly. 

Finally, the article discusses the need for more open models and recommends including the evaluation of performance ratings of earlier prognoses of a model into deciding upon energy strategies.

\section{Methods}
\label{sec:Methods}
For this article, more than 35 sources from 12 different institutions have been reviewed. These can be seen in table \ref{tab:SourcesOfData}.  

To facilitate the compilation, presentation and visualization of the data, a website has been programmed, publicly available at \cite{Hehl2014}

\begin{itemize}
	\item \emph{[include short descriptions of the sources]}
\item description of data-collection, conversion, interpolation
\item short description of mathematical model for performance ratings (mean deviation, time lapse, growth deviation, error as a function of time, etc) 
\item short description of application of performance ratings of \emph{"old"} prognoses to recent prognoses
\end{itemize}

Mention: Since no source has disclosed details as of how the prognoses were made, the forecasting models were assumed to remain constant over time within each affiliated institution. 
This is the closest guess, but can of course be proven wrong by the respective institution by simply publishing the changes of their models between publication dates.

\subsection{Comparing prognoses}

If we define the \emph{horizon of a prognosis} as the largest distance of days between its publication 
 the date of and the prognosis predicts a val
 When comparing two prognoses, the maximum horizon a 
 The array of days is determined by the maximum of days that two  


\begin{table}
		\scriptsize
		\begin{tabular}{ | r |c| c| c| c |}
			\hline
 Source & Type & Page & Year & Institution/Affiliation\\ 
\hline
& & & &\\ 
\citealp{national_energy_board_canada_canadas_2000} & D & 43,44 & 2000 & NEB\\ 
\citealp{td_securities_overview_2002} & T & 19-24 & 2002 & TD Securities\\ 
\citealp{national_energy_board_canada_canadas_2003} & D & 56 & 2003 & NEB\\ 
\citealp{dunbar_oil_2004} & T & 138-150 & 2004 & CERI\\ 
\citealp{alberta_chamber_of_resources_oil_2004} & D & 8 & 2004 & ACR\\ 
\citealp{alberta_energy_and_utilities_board_st98-2005_2005} & D & 4, 2.16 & 2005 & AESRD\\ 
\citealp{capp_canadian_2006} & T & App tabl 1 & 2006 & CAPP\\ 
\citealp{national_energy_board_canada_canadas_2007} & D & 24 & 2007 & NEB\\ 
\citealp{national_energy_board_canada_2009_2009} & D & 20 & 2009 & NEB\\ 
\citealp{national_energy_board_canada_canadas_2011} & T & App A3.31 & 2011 & NEB\\ 
\citealp{millington_canadian_2011} & D & 33, 40 & 2011 & CERI\\ 
\citealp{millington_oil_2011} & D & 3 & 2011 & CERI\\ 
\citealp{capp_crude_2012} & T & App B.1 & 2012 & CAPP\\ 
\citealp{doshi_energy_2012} & D & 26 & 2012 & Citibank\\ 
\citealp{statistics_canada_energy_2012} & T & Tab 4.2-2 & 2012 & Statistics Ca\\ 
\citealp{government_of_alberta_osip_2012} & T &  & 2012 & Government\\ 
\citealp{capp_capps_2012} & T &  & 2012 & CAPP\\ 
\citealp{oilsands_review_oilsands_2012} & T &  & 2012 & OSR\\ 
\citealp{birol_world_2012} & T & 107, 3.5 & 2012 & IEA\\
\hline


		\end{tabular}
	\caption{Sources of data. \emph{Type} will indicate whether the corresponding data was extracted via \textbf{T}ables or \textbf{D}iagrams.}
	\label{tab:SourcesOfData}
\end{table}

\begin{description}
\label{des:DescriptionOfSources}
	\item[NEB] \emph{National Energy Board} \\ Independent federal agency. Purpose: regulate pipelines, energy development and trade in the Canadian public interest.
\item[TD Securities] \emph{TD Bank Group} \\ Provider of capital market products and services to corporate, government and institutional clients.
\item[CERI] \emph{Canadian Energy Research Institute} \\ Independent, non-profit research establishment. The Board consists of a politician and members of: Confer Consulting Ltd., Alberta Department of Energy, Imperial Oil Limited, NEXEN Inc., Natural Resources Canada, OSUM Oil Sands Corporation and the University of Calgary.
\item[ACR] \emph{Alberta Chamber of Resources} \\ Resource based cross-sectoral industry association.
\item[AEUB] \emph{Alberta Energy and Utilities Board} \\ Head of Energy Resources Conservation Board (ERCB),  that regulates the development of Alberta's energy resources.
\item[CAPP] \emph{Canadian Association of Petroleum Producers} \\ Voice of Canada's upstream oil, oil sands and natural gas industry.
\item[Citibank] Provider of financial solutions in corporate and investment banking, credit cards, consumer finance, investment, leasing and private banking.
\item[Statistics Ca] \emph{Statistics Canada} \\ Canada's central statistical office.
\item[Government] \emph{Government of Alberta} \\ Head of Alberta Environment and Sustainable Resource Development. Protects and enhances Alberta's natural environment, ensures  clean and healthy environment.
\item[OSR] \emph{Oilsands Review} \\ Publication on oilsands industry coverage. Reflects interests of oilsands sector and its stakeholders in Canada and around the world including project owners, service and supply companies, investors, governments, communities, First Nations, and environmental groups.

\end{description}


\begin{table}
	\centering
		\begin{tabular}{| c | c | c | c |}
		\hline
Year & Author & Estimate in EJ & Estimate in Gbbl\\
\hline 
& & & \\
1974 & Thomas & 1840 & 298\\ 
1977 & MIT  & 1840 & 298\\ 
1978 & WEC Delphi Poll  & 1360 & 221\\ 
1978 & Ezra & 502 & 81.4\\ 
1981 & Qadeer & 918 & 149\\ 
1982 & Fraas & 1800 & 292\\ 
1985 & Edmonds and Reilly & 3790 & 614\\ 
2002 & Bentley & 4900 & 794\\ 
\hline

		\end{tabular}
	\caption{Estimates of global ultimately recoverable resources of bitumen according to \citet{dale_meta-analysis_2012}. Barrels are calculated assuming an energy content of 6,16  GJ per barrel following \citet{the_oil_drum_net_2008}.}
	\label{tab:EstimatesOfUltimatelyRecoverableResourcesAccordingTo}
\end{table}

Additionally, there are estimates for global oil sands and oil shale resources, given in table \ref{tab:EstimatesOfGlobalOilSandsAndOilShaleResources}. Although the total quantities do not give much information about possible production rates, the picture is clear: Canada, the US and Venezuela have the largest part of global reserves  of  bitumen.

\begin{table}
	\centering
		\begin{tabular}{| l | r | c |}
		\hline
\textbf{Country/Region} & \textbf{Reserves in Gbbl} & \textbf{Remarks} \\ 
\hline
& & \\
Canada & 2520 & Ultimate volume in place \\ 
 & 1630 & Initial volume in place \\ 
 & 308 & Ultimately recoverable \\ 
 & 162 & Proven reserves \\ 
US & 2000 & Ultimate volume in place \\ 
 & 100 & Proven reserves \\ 
Venezuela & 1180 & Volume in place \\ 
 & 235 & Recoverable volume \\ 
Former Soviet Union & 1170 & Initial volume in place \\ 
Morocco & 700 & \\ 
Jordan & 350 & \\ 
Australia & 50& \\ 
Nigeria & 43 & Initial volume in place \\ 
\hline
		\end{tabular}
	\caption{Estimates of global oil sands and oil shale resources according to \citet{oil_and_energy_trends_bitumen_2006}. }
	\label{tab:EstimatesOfGlobalOilSandsAndOilShaleResources}
\end{table}

\section{Discussion}
\label{sec:Discussion}

\begin{itemize}
	\item limitations of the calculations and assumptions within this article
	\item lack of insight into models of respective institutions 
	\begin{itemize}
			\item Why? general bias? Purely industrial interests in the prognoses?
			\item Can this be changed? Need for \emph{open models} that can be peer-reviewed and reproduced!
			\item Appeal to consult performance ratings before relying on a prognosis
	\end{itemize}
	
\end{itemize}


\section{References}
\label{sec:References}
\bibliographystyle{elsarticle-harv}
\bibliography{art_hehl_2014}
 


\end{document}
